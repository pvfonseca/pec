\documentclass[10pt]{beamer}
\usetheme{jambro}

\title[]{Pensamento Econômico Contemporâneo - Escola novo-Keynesiana}
\author[]{Paulo Victor da Fonseca}
\date{}

\hypersetup{
    colorlinks = true,
    urlcolor = teal,
    linkcolor = teal    
}
\usepackage[portuguese]{babel}
\usepackage{subfig}
\usepackage{emoji}

\begin{document}

\begin{frame}[plain]
    \titlepage{
        \begin{center}
            \begin{minipage}{0.8\textwidth}
                \centering
            \end{minipage}
        \end{center}}
\end{frame}

\begin{frame}{Sumário}
    \tableofcontents
\end{frame}

\section{Introdução}
\begin{frame}{Introdução}
    \begin{itemize}
        \item Principais proposições do Keynesianismo ortodoxo:\bigskip
        \begin{enumerate}
            \item uma economia de mercado não regulamentada apresentará períodos prolongados de excesso de oferta de produto e trabalho ($\neq$ lei de Say) - i.e., economias de mercado apresentarão desemprego involuntário \bigskip
            \item instabilidade agregada (ciclos econômicos) é causado, primordialmente, por distúrbios de demanda agregada\bigskip
            \item a moeda importa, na maior parte do tempo, mas em períodos de recessão profunda a política monetária pode ser ineficaz (Blanchard, 1990a; Krugman, 1998)\bigskip
            \item intervenções governamentais sob a forma de políticas de estabilização têm o potencial de melhorar o bem-estar social e a estabilidade macro
        \end{enumerate}
    \end{itemize}
\end{frame}

\begin{frame}{Introdução}
    \begin{figure}
        \centering
        \includegraphics[width=0.8\textwidth]{./figures/aula15_fig1.PNG}
        \caption{Freshwater $\times$ saltwater economics.}
        \label{fig1}
    \end{figure}
\end{frame}

\begin{frame}{Introdução}
    \begin{itemize}
        \item Novos-Keynesianos compartilham das proposições do Keynesianismo ortodoxo, mas modelos novo-Keynesianos são diferentes em muitos aspectos\bigskip
        \item Modelos novo-Keynesianos compartilham duas das premissas metodológicas novo-clássica:\bigskip
        \begin{enumerate}
            \item Teorias macroeconômicas requerem fundamentos microeconômicos sólidos\bigskip
            \item Modelos macroeconômicos construídos no arcabouço de equilíbrio geral.
        \end{enumerate}
        \item Microfundamentos RBC: informação perfeita, competição perfeita, ausência de custos de transação, existência de um conjunto completo de mercados\bigskip
        \item Abordagem novo-Keynesiana reconhece importância de uma variedade de imperfeições, e.g.: informação assimétrica, agentes heterogêneos e mercados imperfeitos e incompletos\bigskip
        \item Agenda de pesquisa com objetivo de retificar falhas teóricas do lado de oferta do modelo Keynesiano ortodoxo - reconstrução dos microfundamentos usando teoria microeconômica moderna\bigskip
        \item As imperfeições incorporadas implicarão que oferta agregada responde a variações de demanda agregada
    \end{itemize}
\end{frame}    

\begin{frame}{Introdução}
    \begin{itemize}
        \item Modelos de tradição clássica: hipótese de equilíbrio contínuo de mercado implica que economia não pode ser restrita por deficiência de demanda agregada\bigskip
        
        \item Modelos Keynesianos: ausência de equilíbrio contínuo de mercados - preços não se ajustam rápido o suficiente para equilibrar mercados e, então, choques de oferta e demanda levam a efeitos reais significativos em produto e emprego\bigskip
        \item Keynesianos ortodoxos $\times$ novos-Keynesianos\bigskip
        \item Portanto, novos-Keynesianos tem como objetivo desenvolver uma teoria coerente de oferta agregada onde rigidezes de preços e salários podem ser racionalizadas:\bigskip
        
        \NB{search for rigorous and convincing models of wage and/or price stickiness based on maximising behaviour and rational expectations

            \flushright{(Gordon, 1990)}}
    \end{itemize}
\end{frame}

\begin{frame}{Introdução}
    \begin{itemize}
        \item Novos-Keynesianos: grupo heterogêneo\bigskip
        \item Gregory Mankiw, Lawrence Summers, Olivier Blanchard, Stanley Fischer, Bruce Greenwald, Edmund Phelps, Joseph Stiglitz, Ben Bernanke, Laurence Ball, George Akerlof, Janet Yellen, David Romer, Robert Hall, John Taylor, Dennis Snower, Assar Lindbeck\bigskip
        \item Europeus: Richard Layard, Stephen Nickell, Richard Jackman, Wendy Carlin, David Soskice\bigskip
        \item Europeus: competição imperfeito no mercado de trabalho e no mercado de bens, refletindo elevadas taxas de sindicalização que caracterizam países europeus
    \end{itemize}
\end{frame}

\begin{frame}{Introdução}
    \begin{itemize}
        \item Explicações (mainstream) alternativas do ciclo econômico do começo dos anos 1980s:\bigskip
        \begin{enumerate}
            \item teorias novo-clássicas de preços flexíveis e percepções monetárias errôneas de Lucas\bigskip
            \item modelos expectacionais de preços rígidos enfatizando algum elemento de rigidez de salários e preços (Fischer, 1977; Phelps e Taylor, 1977; Taylor, 1980)\bigskip
            \item teoria RBC
        \end{enumerate}
    \end{itemize}
\end{frame}

\begin{frame}{Economia novo-Keynesiana}
    \begin{itemize}
        \item Mankiw e Romer (1991):
        \begin{enumerate}
            \item A teoria viola a dicotomia clássica? Isto é, a moeda é não-neutra?\bigskip
            \item A teoria assume que imperfeições reais de mercado são cruciais para a compreensão das flutuações econômicas?\bigskip
        \end{enumerate}
        \item Escola novo-Keynesiana: resposta afirmativa para ambas as perguntas\bigskip
        \item Não-neutralidade emerge da rigidez de preços, imperfeições de mercado explicam este comportamento dos preços\bigskip
        \item A interação entre imperfeições reais e nominais distinguem a economia novo-Keynesiana (Mankiw e Romer, 1991)
    \end{itemize}    
\end{frame}

\begin{frame}{Economia novo-Keynesiana}
    \begin{itemize}
        \item Novos-Keynesianos não são protagonistas no debate Keynesianos-monetaristas da década de 1960:\bigskip
        \begin{enumerate}
            \item Não há uma visão unificada sobre o papel da política fiscal - novos-Keynesianos dão um peso muito maior sobre o papel estabilizador da política monetária (nova economia monetária)\bigskip
            \item Não há uma visão unificada acerca da desejabilidade e viabilidade de políticas ativistas (discricionárias) de estabilização
        \end{enumerate}
    \end{itemize}
\end{frame}

\begin{frame}{Economia novo-Keynesiana}
    \begin{itemize}
        \item Algumas características:\bigskip
        \begin{enumerate}
            \item Em contraste com modelos novo-clássicos habitados por agentes tomadores de preços, modelos novo-Keynesianos assumem firmas fixadoras de preços em um ambiente de competição monopolística (Robinson, 1933; Chamberlin, 1933)\bigskip
            \item A maioria dos modelos novo-Keynesianos assume que expectativas são formadas racionalmente (alguns permanecem críticos com relação aos fundamentos teóricos e suporte empírico da HER)\bigskip
            \item Choques de demanda e oferta como potenciais fontes de instabilidade (Blanchard e Quah, 1989)\bigskip
            \item Muitos (não todos) compartilham a visão Keynesiana de que desemprego involuntário é possível e provável\bigskip
        \end{enumerate}
    \end{itemize}
\end{frame}

\begin{frame}{Economia novo-Keynesiana}
    \begin{itemize}
        \item Grande heterogeneidade entre novos-Keynesianos: conveniente dividir as explicações de rigidezes entre aquelas que focam em rigidezes nominais e as que focam em rigidezes reais\bigskip
        \item Rigidez nominal: quando algo impede o ajuste perfeito do nível geral de preços frente a distúrbios de demanda nominal\bigskip
        \item Rigidez real: algum fator impede ajuste de salários reais ou se há alguma rigidez de um salário relativo a outro, ou de um preço relativo a outro
    \end{itemize}
\end{frame}

\section{Rigidezes nominais}
\begin{frame}{Rigidez nominal}
    \begin{itemize}
        \item Em linha com a abordagem de teoria da escolha da análise novo-clássica, a abordagem NK assume que trabalhadores e firmas são maximizadores de utilidade e lucros, respectivamente\bigskip
        \item Novos-clássicos: modelo de leilões de preços flexíveis adotado e aplicado à análise das transações realizadas em todos os mercados (incluindo mercado de trabalho)\bigskip
        \item Novos-Keynesianos: distinção entre mercados que são essencialmente de preços fixos (predominantemente o mercado de trabalho e boa parte dos mercados de bens) e mercados de preços flexíveis (mercados financeiros e de commodities)\bigskip
        \item Mercados de preços fixos: a norma é fixação de preços, com inércia de preços e salários uma realidade\bigskip
        \item Não-neutralidade monetária é gerada pela incapacidade de preços e salários de se ajustar instantaneamente ao novo nível de equilíbrio de mercado após um choque de demanda agregada
    \end{itemize}
\end{frame}

\begin{frame}{Rigidez nominal}
    \begin{itemize}
        \item Keynesianos tradicionalmente concentraram-se no mercado de trabalho e em rigidezes de salário nominal\bigskip
        \item No entanto, é importante observar que para qualquer dada trajetória de demanda agregada nominal, é a rigidez de preços, e não de salários, que é necessária para gerar flutuações no produto agregado real\bigskip
        \item Mas a `primeira onda' de modelos novo-Keynesianos focou em rigidez de salários nominais
    \end{itemize}
\end{frame}

\begin{frame}{Rigidez de salário nominal}
    \begin{itemize}
        \item Modelo Keynesiano tradicional: nível de preços impedido de cair para restaurar equilíbrio pela incapacidade de salários nominais (custos) em se ajustarem - Figura \ref{fig2}\bigskip
        \item Modelos novo-clássicos: choques antecipados de política monetária levam a uma redução imediata de salários nominais e preços para valor de equilíbrio, mantendo produto e emprego agregados\bigskip
        \item Ineficácia de política monetária: resultado da hipótese de expectativas racionais\bigskip
        \item Fischer (1977) e Phelps e Taylor (1977): choques nominais produzem efeitos reais em modelos com HER se a hipótese de equilíbrio contínuo de mercado for abandonada\bigskip
        \item \hlight{Hipótese de expectativas racionais não implica o fim da economia Keynesiana}
    \end{itemize}
\end{frame}

\begin{frame}{Rigidez de salário nominal}
    \begin{figure}
        \centering
        \includegraphics[width=0.3\textwidth]{./figures/aula5_fig1.PNG}
        \caption{Desemprego involuntário - modelo Keynesiano tradicional.}
        \label{fig2}
    \end{figure}
\end{frame}

\begin{frame}{Rigidez de salário nominal}
    \begin{itemize}
        \item Fischer (1977) e Taylor (1980): introdução de inércia nominal sob a forma de contratos salariais de longo prazo\bigskip
        \item Salários não são determinados em mercados à vista (spot markets) mas tendem a ser fixados por um período de tempo acordado sob a forma de um contrato explícito (ou implícito)\bigskip
        \item A existência de contratos de longo prazo pode gerar suficiente rigidez de salário nominal para que política monetária reestabeleça sua efetividade\bigskip
        \item Note, no entanto, que nem Fischer nem Phelps e Taylor forneceram um microfundamento rigoroso para as hipóteses de fixação de preços e salários\bigskip
        \item Ao invés disso, há uma ``preferência revelada'' por contratos salariais de longo prazo refletindo as desvantagens percebidas de incorrer em ajustes frequentes de preços e salários
    \end{itemize}
\end{frame}

\begin{frame}{Rigidez de salário nominal}
    \begin{itemize}
        \item Fischer: modelo com estrutura similar ao de Lucas-Sargent-Wallace\bigskip
        \item Função de oferta agregada de Lucas:
        \begin{equation}
            Y_t = Y_t^n + \alpha(\dot{P}_t - \dot{P}_t^e), \qquad \alpha > 0
        \end{equation}
        \item Com expectativas formadas racionalmente:
        \begin{equation}
            Y_t = Y_t^n + \alpha\left[\dot{P}_t - \mathbb{E}(\dot{P}_t|\Omega_{t-1})\right]
        \end{equation}
        \item Modelo assume que não há crescimento, portanto, assume-se que negociadores salariais buscam constância de salário real fixando aumentos de salários nominais iguais à expectativa inflacionária:
        \begin{equation}
            \dot{W}_t = \mathbb{E}(\dot{P}_t|\Omega_{t-1})
        \end{equation}
    \end{itemize}
\end{frame}

\begin{frame}{Rigidez de salário nominal}
    \begin{itemize}
        \item Portanto, oferta agregada é uma função inversa do salário real (\hlight{salário real é contracíclico}):
        \begin{equation}
            Y_t = Y_t^n + \alpha[\dot{P}_t - \dot{W}_t] 
        \end{equation}
        \item Para o contratos multi-períodos, aumentos dos salários nominais são fixados em $\dot{W}_t = \dot{W}_t^*$\bigskip
        \item Fischer (1977) assume (`hipótese empiricamente razoável') que agentes econômicos negociam salários em termos nominais para períodos mais longos do que o tempo que leva para a autoridade monetária reagir ao ambiente econômico\bigskip
        \item Como autoridades monetárias podem alterar a oferta de moeda (e, portanto, inflação) de maneira mais frequente que contratos de trabalho sobrepostos são renegociados, a política monetária pode ter efeitos reais no curto prazo (mantendo-se neutra no longo prazo)
    \end{itemize}
\end{frame}

\begin{frame}{Rigidez de salário nominal}
    \begin{figure}
        \centering
        \includegraphics[width=0.6\textwidth]{./figures/aula15_fig2.PNG}
        \caption{Contratos de salários nominais, expectativas racionais e política monetária. Fonte: Snowdon e Vane (2005).}
        \label{fig3}
    \end{figure}
\end{frame}

\begin{frame}{Rigidez de salário nominal}
    \begin{itemize}
        \item Economia operando ponto A\bigskip
        \item Suponha um choque não-antecipado de demanda agregada (e.g., queda na velocidade) - desloca curva de $AD_0$ para $AD_1$\bigskip
        \item Se preços são flexíveis mas salários nominais temporariamente rígidos (fixados em $W_0$) como resultado dos contratos negociados no período anterior que se estendem para além do período corrente, então a economia se move para o ponto B (produto agregado real contrai de $Y_N$ para $Y_1$)\bigskip
        \item Se preços e salários são flexíveis, curva de OA de curto prazo deslocaria para a direita - $SRAS(W_0)$ para $SRAS(W_1)$, reestabelecendo produto natural ponto C\bigskip
        \item A existência de contratos salariais nominais de longo prazo impedem que isso aconteça e dá oportunidade para que autoridade monetária expanda a oferta de moeda que, mesmo que antecipado, desloca DA para a direita reestabelecendo equilíbrio no ponto A
    \end{itemize}
\end{frame}

\begin{frame}{Rigidez de salário nominal}
    \begin{itemize}
        \item Caso autoridades consigam reagir a choques exógenos numa frequência maior que setor privado possa renegociar salários nominais há espaço para gerenciamento de DA para estabilizar a economia (mesmo que agentes formem expectativas racionalmente)\bigskip
        \item A não-neutralidade da moeda no modelo de Fischer não se deve a uma surpresa monetária não-antecipada\bigskip
        \item Políticas monetárias antecipadas têm efeitos reais porque são baseadas em informações que só se tornam disponíveis depois que o contrato foi firmado
    \end{itemize}
\end{frame}

\begin{frame}{Rigidez de salário nominal}
    \begin{itemize}
        \item Contratos salariais são características importante em quase todas economias desenvolvidas: heterogeneidade com relação à duração de contratos e timing de renegociação\bigskip
        \item Japão: contratos de salários nominais tipicamente duram 1 ano e expiram simultaneamente. Renegociações sincronizadas (sistema \emph{shunto}) é consistente com maior estabilidade macro do que em países que apresentam contratos não-sincronizados (sobrepostos)\bigskip
        \item EUA: contratos sobrepostos/não-sincronizados tipicamente de 3 anos (Gordon, 1982b; Hall e Taylor, 1997)\bigskip
        \item UK: contratos não-sincronizados com duração de 1 ano
    \end{itemize}
\end{frame}

\begin{frame}{Rigidez de salário nominal}
    \begin{itemize}
        \item Com contratos sobrepostos, salários nominais exibirão mais inércia frente a choques do que seria o caso com renegociações sincronizadas\bigskip
        \item Taylor (1980): se trabalhadores estão preocupados com seus salários nominais relativos a outros, então, contratos sobrepostos permitirão que os impactos da política monetária sobre variáveis reais persistam por um tempo bem maior do que a extensão do período de contrato\bigskip
        \item Taylor (1992b): a responsividade de salários a choques de oferta e demanda é maior no Japão que nos EUA, Canadá e outros países europeus, e isso possibilitou uma maior estabilidade macro no Japão durante os anos 1970s e 1980s
    \end{itemize}
\end{frame}

\begin{frame}{Rigidez de salário nominal}
    \begin{itemize}
        \item Por que contratos de longo prazo são formados se aumentam instabilidade macro? Phelps (1985, 1990) argumenta que há razões privadas para firmas e trabalhadores:\bigskip
        \begin{enumerate}
            \item Negociações salariais são custosas para firmas e trabalhadores: pesquisas com relação à estrutura de relatividades dentro e fora da organização; necessidade de formar previsões p/ variáveis relevantes (produtividade, inflação, demanda, lucros e preços). Quanto maior o período do contrato, menos frequentemente estes custos de transação serão incorridos\bigskip
            \item Sempre existe potencial para não haver um acordo: trabalhadores podem recorrer a greves para reforçar posição no processo de barganha (disrupções custosas tanto para firmas quanto trabalhadores)\bigskip
            \item Não será uma estratégia ótima para uma firma ajustar seus salários para o novo valor de equilíbrio frente a um choque adverso de demanda porque, caso outras firmas não o façam, a firma reduziria seu salário relativo e isso, por sua vez, poderia aumentar a rotatividade do trabalho
        \end{enumerate}
    \end{itemize}
\end{frame}

\begin{frame}{Rigidez de salário nominal}
    \begin{itemize}
        \item Por que contratos salariais não são indexados à taxa de inflação?\bigskip
        \item Acordos completos de custo de vida (COLAs) são arriscados para as firmas pois nem todos os choques são choques de demanda nominal\bigskip
        \item Se uma firma indexa seus salários à inflação então, choques de oferta (como os dos anos 1970s) levariam a um aumento no nível de preços e, com isso, a um aumento nos custos salariais das firmas e, portanto, impedindo a queda necessária de salário real implicada por um choque de energia
    \end{itemize}
\end{frame}

\begin{frame}{Rigidez de salário nominal}
    \begin{itemize}
        \item Contratos salariais não-sincronizados também podem ter propósitos microeconômicos, mesmo que causem instabilidade macro\bigskip
        \item Se firmas tem informação imperfeita a respeito da situação econômica corrente, podem obter informações relevantes ao observar preços e salários fixados por outras firmas\bigskip
        \item Hall e Taylor (1997): fixação sobrepostas de salários provê informações úteis para firmas e trabalhadores a respeito de variações na estrutura de preços e salários\bigskip
        \item Em um sistema decentralizado sem acordos sobrepostos, observaríamos uma enorme variabilidade introduzida no sistema\bigskip
        \item Ball e Cecchetti (1988): informação imperfeita pode tornar estratégias de salários e preços sobrepostos socialmente ótimos ao ajudar firmas a fixar preços próximo aos níveis de informação perfeita, levando a ganhos de eficiência que mais que compensam os custos de inércia no nível de preços\bigskip
        \item \hlight{Ajustes salariais sobrepostos podem emergir do comportamento racional dos agentes. Fixação de salários em um sistema sincronizado parece requerer algum grau de participação ativa do governo} 
    \end{itemize}
\end{frame}

% \begin{frame}{\emoji{books} Bibliografia}
%     \begin{itemize}                        
%         \item Solow, R.M. (1998), ‘How Cautious Must the Fed Be?’, in R.M. Solow and J.B. Taylor, Inflation, Unemployment and Monetary Policy, Cambridge, MA: MIT Press\medskip        
%         \item Svensson, L.E.O. (1997), ‘Optimal Inflation Targets, “Conservative” Central Banks and Linear Inflation Contracts’, American Economic Review\medskip
%         \item Taylor, H. (1985), ‘Time Inconsistency: A Potential Problem for Policymakers’, Federal Reserve Bank of Philadelphia Business Review\medskip
%         \item Taylor, J.B. (1980), ‘Aggregate Dynamics and Staggered Contracts’, Journal of Political Economy\medskip
%         \item Waller, C.J. and Walsh, C.E. (1996), ‘Central Bank Independence, Economic Behaviour and Optimal Term Lengths’, American Economic Review\medskip
%         \item Walsh, C.E. (1993), ‘Central Bank Strategies, Credibility and Independence: A Review Essay’, Journal of Monetary Economics\medskip
%         \item Walsh, C.E. (1995), ‘Optimal Contracts for Central Bankers’, American Economic Review\medskip
%     \end{itemize}
% \end{frame}
\end{document}